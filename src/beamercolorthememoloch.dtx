% \iffalse meta-comment -------------------------------------------------------
% Copyright 2015 Matthias Vogelgesang and the LaTeX community. A full list of
% contributors can be found at
%
%     https://github.com/matze/mtheme/graphs/contributors
%
% and the original template was based on the HSRM theme by Benjamin Weiss.
%
% This work is licensed under a Creative Commons Attribution-ShareAlike 4.0
% International License (https://creativecommons.org/licenses/by-sa/4.0/).
%% --------------------------------------------------------------------------- 
%% Copyright 2024 Johan Larsson and contributors
% ------------------------------------------------------------------------- \fi
% \iffalse
%<*package>
\NeedsTeXFormat{LaTeX2e}
\ProvidesPackage{beamercolorthememoloch}[2024-07-09 v0.4.0 Moloch color theme] % x-release-please-version
%</package>
% \fi
% \CheckSum{0}
% \StopEventually{}
% \iffalse
%<*package>
% ------------------------------------------------------------------------- \fi
%
% \subsection{\themename color theme}
%
%
%
% \subsubsection{Package dependencies}
%    \begin{macrocode}
\RequirePackage{pgfopts}
%    \end{macrocode}
%
%
%
% \subsubsection{Options}
%
% \begin{macro}{block}
%    Optionally adds a light grey background to block environments like
%    |theorem| and |example|.
%    \begin{macrocode}
\pgfkeys{
  /moloch/color/block/.cd,
  .is choice,
  transparent/.code=\moloch@block@transparent,
  fill/.code=\moloch@block@fill,
}
%    \end{macrocode}
% \end{macro}
%
% \begin{macro}{colors}
%    Provides the option to have a dark background and light foreground instead
%    of the reverse.
%    \begin{macrocode}
\pgfkeys{
  /moloch/color/background/.cd,
  .is choice,
  dark/.code=\moloch@colors@dark,
  light/.code=\moloch@colors@light,
}
%    \end{macrocode}
% \end{macro}
%
% \begin{macro}{\moloch@color@setdefaults}
%    Sets default values for color theme options.
%    \begin{macrocode}
\newcommand{\moloch@color@setdefaults}{
  \pgfkeys{/moloch/color/.cd,
    background=light,
  }
}
%    \end{macrocode}
% \end{macro}
%
%
%
% \subsubsection{Base colors}
%
%    \begin{macrocode}
\definecolor{mDarkBrown}{HTML}{604c38}
\definecolor{mDarkTeal}{HTML}{23373b}
\definecolor{mLightBrown}{HTML}{EB811B}
\definecolor{mLightGreen}{RGB}{0,128,128}
%    \end{macrocode}
%
%
%
% \subsubsection{Base styles}
%
% All colors in \themename are derived from the definitions of |normal text|,
% |alerted text|, and |example text|.
%
%    \begin{macrocode}
\newcommand{\moloch@colors@dark}{
  \setbeamercolor{normal text}{%
    fg=black!2,
    bg=mDarkTeal
  }
  \usebeamercolor[fg]{normal text}
}
\newcommand{\moloch@colors@light}{
  \setbeamercolor{normal text}{%
    fg=mDarkTeal,
    bg=black!2
  }
}
\setbeamercolor{alerted text}{%
  fg=mLightBrown
}
\setbeamercolor{example text}{%
  fg=mLightGreen
}
%    \end{macrocode}
%
%
%
% \subsubsection{Derived colors}
%
% The titles and structural elements (e.g. |itemize| bullets) are set in the
% same color as |normal text|. This would ideally done by setting |normal text|
% as a parent style, which we do to set |titlelike|, but this doesn't work for
% |structure| as its foreground is set explicitly in
% |beamercolorthemedefault.sty|.
%
%    \begin{macrocode}
\setbeamercolor{titlelike}{use=normal text, parent=normal text}
\setbeamercolor{author}{use=normal text, parent=normal text}
\setbeamercolor{date}{use=normal text, parent=normal text}
\setbeamercolor{institute}{use=normal text, fg=normal text.fg!80!normal text.bg}
\setbeamercolor{structure}{use=normal text, fg=normal text.fg}
%    \end{macrocode}
%
% The “primary” palette should be used for the most important navigational
% elements, and possibly of other elements. \themename uses it for frame
% titles and slides.
%
%    \begin{macrocode}
\setbeamercolor{palette primary}{%
  use=normal text,
  fg=normal text.bg,
  bg=normal text.fg
}
\setbeamercolor{frametitle}{%
  use=palette primary,
  parent=palette primary
}
%    \end{macrocode}
%
% The \themename inner or outer themes optionally display progress
% bars in various locations. Their color is set by |progress bar| but the two
% different kinds can be customized separately. The horizontal rule on the
% title page is also set based on the progress bar color and can be customized
% with |title separator|.
%
%    \begin{macrocode}
\setbeamercolor{progress bar}{%
  use=alerted text,
  fg=alerted text.fg,
  bg=alerted text.fg!50!black!30
}
\setbeamercolor{title separator}{
  use=progress bar,
  parent=progress bar
}
\setbeamercolor{progress bar in head/foot}{%
  use=progress bar,
  parent=progress bar
}
\setbeamercolor{progress bar in section page}{
  use=progress bar,
  parent=progress bar
}
%    \end{macrocode}
%
% Block environments such as |theorem| and |example| have no background color
% by default. The option |block=fill| sets a background color based on the
% background and foreground of |normal text|. The option |block=transparent|
% reverts the block environments to an empty background, which can be useful
% if changing colors mid-presentation.
%
%    \begin{macrocode}
\newcommand{\moloch@block@transparent}{
  \setbeamercolor{block title}{bg=}
  \setbeamercolor{block body}{bg=}
  \setbeamercolor{block title alerted}{bg=}
  \setbeamercolor{block title example}{bg=}
}
\newcommand{\moloch@block@fill}{
  \setbeamercolor{block title}{%
    bg=normal text.bg!80!fg
  }
  \setbeamercolor{block body}{%
    use=block title,
    bg=block title.bg!50!normal text.bg
  }
  \setbeamercolor{block title alerted}{%
    bg=block title.bg,
  }
  \setbeamercolor{block title example}{%
    bg=block title.bg,
  }
}
\setbeamercolor{block title}{%
  use=normal text,
  fg=normal text.fg
}
\setbeamercolor{block title alerted}{%
  use={block title, alerted text},
  fg=alerted text.fg
}
\setbeamercolor{block title example}{%
  use={block title, example text},
  fg=example text.fg
}
\setbeamercolor{block body alerted}{use=block body, parent=block body}
\setbeamercolor{block body example}{use=block body, parent=block body}
%    \end{macrocode}
%
% Footnotes
%
%    \begin{macrocode}
\setbeamercolor{footnote}{fg=normal text.fg!90}
\setbeamercolor{footnote mark}{fg=.}
%    \end{macrocode}
%
% We also reset the bibliography colors in order to pick up the surrounding
% colors at the time of use. This prevents us having to set the correct color in
% normal and standout mode.
%
%    \begin{macrocode}
\setbeamercolor{bibliography entry author}{fg=, bg=}
\setbeamercolor{bibliography entry title}{fg=, bg=}
\setbeamercolor{bibliography entry location}{fg=, bg=}
\setbeamercolor{bibliography entry note}{fg=, bg=}
%    \end{macrocode}
%
%
%
% \subsubsection{Process package options}
%
%    \begin{macrocode}
\moloch@color@setdefaults
\ProcessPgfPackageOptions{/moloch/color}
%    \end{macrocode}
%
%    \begin{macrocode}
\mode<all>
%    \end{macrocode}
%
% \iffalse
%</package>
% \fi
% \Finale
\endinput
