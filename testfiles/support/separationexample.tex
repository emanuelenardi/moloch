\input regression-test

\title{Moloch Test Suite}
\subtitle{A subtitle that is way to long and in fact might just need to be split across lines}

\author[Johan]{Johan Larsson}
\institute[LU]{Lund Univesity//Department of Statistics}
\date{April 23, 2024}

\begin{document}

\START
\showoutput

\begin{frame}{Table of contents}
  \setbeamertemplate{section in toc}[sections numbered]
  \tableofcontents[hideallsubsections]
\end{frame}

\section{Results}

\subsection{Proof of the Main Theorem}

\begin{frame}<1>
  \frametitle{There Is No Largest Prime Number}
  \framesubtitle{The proof uses \textit{reductio ad absurdum}.}

  \begin{theorem}
    There is no largest prime number.
  \end{theorem}
  \begin{proof}
    \begin{enumerate}
      \item<1-| alert@1> Suppose $p$ were the largest prime number.
      \item<2-> Let $q$ be the product of the first $p$ numbers.
      \item<3-> Then $q$\;+\,$1$ is not divisible by any of them.
      \item<1-> Thus $q$\;+\,$1$ is also prime and greater than $p$.\qedhere
    \end{enumerate}
  \end{proof}
\end{frame}

\vfil\break
\END

\end{document}
